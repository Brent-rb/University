\documentclass[a4paper]{article}

\usepackage[dutch]{babel}
\usepackage[utf8]{inputenc}
\usepackage{amsmath}
\usepackage{graphicx}
\usepackage[colorinlistoftodos]{todonotes}
\usepackage{enumitem}

\title{Professionele content versus user-generated content.}

\author{Brent Berghmans 1334252}

\date{\today}

\begin{document}
\maketitle
\begin{abstract}De groei van het internet heeft ook een groei van user-generated content (UGC) met zich mee gebracht. User-generated content is de verzamelnaam voor allerlei media die gemaakt wordt door gebruikers van een online systeem of service (Blogs, YouTube video’s, beoordelingen op webshops, …). In deze paper gaan we de focus leggen op een deelverzameling van UGC namelijk reviews, nieuwsartikelen etc. en in het bijzonder de betrouwbaarheid van deze content in verhouding met professionele content waarmee we al jaren vertrouwd zijn. Hiernaast bespreken we kort waarom de betrouwbaarheid zo belangrijk is en de populariteit van UGC.
\end{abstract}

\section{Populariteit}
De opkomst van webshops heeft als gevolg een nood aan online reviews met zich mee gebracht. Online reviews kunnen we onderverdelen in 2 categorieën: User-generated reviews (UGR) en professionele reviews (PR). 
Uit artikel [1] blijkt dat potentiële kopers een voorkeur hebben aan UGR, hoe komt dit? 
Hoewel hier nog geen studies op gedaan zijn kunnen er wel wat redenen voor bedacht worden:
\begin{itemize}
\item Ze voelen oprechter aan, PR kunnen soms wat afstandelijker aanvoelen.
\item Afhankelijk van het product kunnen sommige reviews geschreven zijn na een langdurig gebruik, professionele reviewers hebben meestal maar even de tijd voor ze het product moeten terugsturen. Of ze kunnen het product niet lang gebruiken voordat ze een ander product moeten reviewen.
\item Er is een zeer groot aanbod UGR. (eventueel het enige aanbod bij niet zo’n populaire producten.)
\end{itemize}
\newpage
Per definitie is iedereen toegelaten om een user-generated review te schrijven, dit brengt natuurlijk een aantal gevaren met zich mee:
\begin{itemize}
\item Een bedrijf kan iemand betalen om een slechte review over een concurrerend product of een goede review over hun product te schrijven.
\item Deze reviews zijn soms wat meer subjectief.
\item Men durft eventueel geen hele slechte scores te geven.
\item Tegenwoordig moet men overal ook oppassen voor “trollen”.\footnote{Een trol (mv.: trollen) in een internetomgeving is een persoon die berichten plaatst met het doel voorspelbare emotionele reacties (bijvoorbeeld woede, irritatie, verdriet, of scheldpartijen) van andere mensen uit te lokken, opzettelijk verkeerde informatie (desinformatie) te geven of zichzelf expres anders voor te doen en een rol te spelen.[3]}
\item Degene die de review heeft geschreven is misschien niet zo ervaren met de categorie van het product (vb.: Grafische kaarten, smartphones, processoren…).
\end{itemize}
Gelukkig weten de meeste kopers dit ook, we zien dat 80\% van de potentiële kopers dan ook sceptisch zijn over de betrouwbaarheid van een user-generated review [1], men bekijkt dan ook eerst gemiddeld 11 reviews voordat men een beslissing maakt, dit zien we ook terug in de vorm van “wisdom of the crowd” [2].
\section{Betrouwbaarheid}
Artikel [2] bekijkt in detail de betrouwbaarheid van UGR van hotels in Singapore op de website TripAdvisor. De resultaten zijn dat deze reviews vrij betrouwbaar zijn en dat de uitleg bij de review meestal overeenkomt met de toegevoegde score. Hieruit kunnen we al concluderen dat men in het algemeen geen schrik moet hebben dat reviewers geen slechte score durven geven.\\\\
We zouden kunnen concluderen dat UGR ongeveer even betrouwbaar zijn als PR maar ik denk dat dit sterk afhangt van de review materie. UGR voor films zijn waarschijnlijk een stuk subjectiever dan PR maar langs de andere kant kunnen bepaalde films een slechte score krijgen door professionele reviewers maar een goede score door user reviewers (wegens meningen), en het kan goed zijn dat jou mening eerder overeenkomt met de meerderheid van de UGR waardoor deze UGR relevanter zijn voor jou dan de PR.\\\\
Als we dan bijvoorbeeld producten uit de elektronica wereld nemen, waar prestaties de ervaring mee bepalen dan is het al moeilijker om een groot verschil te vinden tussen UGR en PR.\\
Als conclusie kunnen we dus stellen dat de betrouwbaarheid van UGR te bepalen zijn aan de hand van het aantal reviews dat beschikbaar zijn over het product en wat voor soort product het is. Uit de meegeleverde artikels lijkt het alsof UGR minstens even betrouwbaar zijn als PR maar er is meer onderzoek nodig voor men dit kan veralgemeniseren naar alle UGC.
\newpage
\section{Toekomst}
Ik denk dat er nog veel onderzoek gedaan moet worden voordat men met zekerheid kan zeggen dat UGC even betrouwbaar is als de “standaard” professionele content die we al jaren gewoon zijn. Hiernaast is er zeker nog onderzoek nodig naar de link tussen de betrouwbaarheid van UGC en de soort van content waarover het gaat.
\newpage
REFERENCES
\begin{enumerate}[label={[\arabic*]}]
  \item Williams, Bradford. Buy It, Try It, Rate It. S.L: Weber Shandwick, S.D, 10
  \item Alton Y. K. Chua and Snehasish Banerjee. Reliability of Reviews on the Internet: The Case of TripAdvisor. San Francisco USA, S.E, 23-25 Oktober 2013, 5
  \item Wikipedia: Trol. http://nl.wikipedia.org/wiki/Trol\_\%28internet\%29
\end{enumerate}
\end{document}